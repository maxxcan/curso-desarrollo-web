% Created 2017-11-07 mar 12:15
% Intended LaTeX compiler: pdflatex
\documentclass[presentation]{beamer}
\usepackage[utf8]{inputenc}
\usepackage[T1]{fontenc}
\usepackage{graphicx}
\usepackage{grffile}
\usepackage{longtable}
\usepackage{wrapfig}
\usepackage{rotating}
\usepackage[normalem]{ulem}
\usepackage{amsmath}
\usepackage{textcomp}
\usepackage{amssymb}
\usepackage{capt-of}
\usepackage{hyperref}
\usepackage{minted}
\usepackage{mathpazo}
\usepackage{stmaryrd}
\DeclareMathOperator*{\argmin}{arg\,min}
\DeclareMathOperator*{\argmax}{arg\,max}
\usetheme{Madrid}
\author{Patricio Martínez}
\date{\textit{<2017-11-07 mar>}}
\title{presentacion-introduccion-javascript}
\hypersetup{
 pdfauthor={Patricio Martínez},
 pdftitle={presentacion-introduccion-javascript},
 pdfkeywords={},
 pdfsubject={},
 pdfcreator={Emacs 27.0.50 (Org mode 9.1.2)}, 
 pdflang={English}}
\begin{document}

\maketitle


\begin{frame}[label={sec:orgd428ebe}]{Introducción a JavaScript}
\begin{block}{Por qué JavaScript}
Junto con HTML y CSS es lenguaje más importante para desarrollo web

\begin{center}
\includegraphics[width=.9\linewidth]{./img/html-css-js-logos.png}
\end{center}
\end{block}
\end{frame}
\begin{frame}[label={sec:org337c04c}]{Características}
\begin{enumerate}
\item Se ejecuta en el lado del cliente. Aunque también se puede ejecutar en el lado del servidor pero no es lo habitual.
\item Lenguaje interpretado. El navegador interpreta el lenguaje si necesidad de compilador
\item Lenguaje basado en objetos.
\item Lenguaje Imperativo y estructurado. Usa las estructuras básicas
\item Lenguaje Objetual, con tipificación débil y dinámica.
\end{enumerate}
\end{frame}
\begin{frame}[label={sec:org3f69a09}]{El curso}
\begin{itemize}
\item Para principiantes
\item Con ejemplos prácticos
\item Podréis bajaros el código desde mi github
\item Está basado en el tutorial de JavaScript de W3CSchools
\end{itemize}
\end{frame}
\begin{frame}[label={sec:org6cf87c6}]{Contacto}
\alert{Os podréis poner en contacto conmigo a través de mi correo o a través de twitter}

\begin{itemize}
\item Correo: maxxcan@gmail.com
\item twitter: @maxxcan
\end{itemize}
\end{frame}
\begin{frame}[label={sec:orgfa1af5d}]{}
\end{frame}
\end{document}
