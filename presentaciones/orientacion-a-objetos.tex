% Created 2018-04-16 lun 18:54
% Intended LaTeX compiler: pdflatex
\documentclass[t]{beamer}
\usepackage[utf8]{inputenc}
\usepackage[T1]{fontenc}
\usepackage{graphicx}
\usepackage{grffile}
\usepackage{longtable}
\usepackage{wrapfig}
\usepackage{rotating}
\usepackage[normalem]{ulem}
\usepackage{amsmath}
\usepackage{textcomp}
\usepackage{amssymb}
\usepackage{capt-of}
\usepackage{hyperref}
\usetheme{Madrid}
\usepackage{mathpazo}
\usepackage{stmaryrd}
\DeclareMathOperator*{\argmin}{arg\,min}
\DeclareMathOperator*{\argmax}{arg\,max}
\usetheme{default}
\author{Patricio Martínez}
\date{\textit{<2017-11-14 mar>}}
\title{Programación orientada a objetos}
\hypersetup{
 pdfauthor={Patricio Martínez},
 pdftitle={Programación orientada a objetos},
 pdfkeywords={},
 pdfsubject={},
 pdfcreator={Emacs 27.0.50 (Org mode 9.1.9)}, 
 pdflang={English}}
\begin{document}

\maketitle
\begin{frame}[label={sec:orgaa21e36}]{Introducción}
A lo largo de la historia han habido varias evoluciones en la informática que la ha mejorada mucho hasta nuestros días:

Esta evolución ha sido gracias a:

\begin{itemize}
\item Avances tecnológicos
\item Avances conceptuales
\item Avances en cuanto al enfoque de la programación
\end{itemize}
\end{frame}

\begin{frame}[label={sec:org9f73577}]{Un poco de historia}
\alert{Simula 67}

Se desarrolló en el año \alert{1967} pero no se reconoció hasta tiempo después. 

Inspirados en el tenemos:

\begin{itemize}
\item Smalltak (1969- 1972)
\item C++ (1983)
\item Y muchos otros
\end{itemize}

\begin{block}{Tipos de programación}
\begin{itemize}
\item Programación lineal
\item Programación estructurada
\item Programación funcional
\item Programación orientada a objetos
\end{itemize}
\end{block}
\end{frame}

\begin{frame}[label={sec:org5dddb92}]{Qué es la POO (OOP)}
Es un conjunto de técnicas que nos permiten incrementar enormemente nuestro proceso de producción de software aumentando drásticamente nuestra productividad y permitiéndonos además abordar proyectos de mucha mayor envergadura.

Usando estas técnicas, nos aseguramos la re-usabilidad de nuestro código.
\end{frame}

\begin{frame}[label={sec:org5f2757c}]{Características y conceptos}
\begin{block}{Definición de Clase}
Es una abstracción que hacemos de nuestra experiencia sensible. Una \alert{clase define} las propiedades y métodos necesarios para \alert{crear objetos}.

También tenemos las \alert{superclases y clases virtuales o abstractas}. 
\end{block}

\begin{block}{Definición de Objeto}
Un objeto es un conjunto de datos y métodos. Lo importante es que ambos están intrínsecamente ligados y forman una única unidad conceptual y operacional.
\end{block}
\end{frame}



\begin{frame}[label={sec:orgb4462f1}]{Herencia}
Esta es una de las características más importantes ya que nos permite reutilizar mucho código o incluso usar código creado por otra persona. 

Dos palabras claves son \alert{this} y \alert{super} 
\end{frame}

\begin{frame}[label={sec:org8ef1bf3}]{Encapsulación}
Llamamos encapsulación al \alert{intrínseco vínculo entre datos y métodos} y al modo de acceder y modificar las propiedades. 
\end{frame}

\begin{frame}[label={sec:org38ca51c}]{Polimorfismo}
El polimorfismo es la capacidad de los objetos de ante el mismo mensaje responder de distinto modo
\end{frame}

\begin{frame}[label={sec:org7b4eeb5}]{Sobrecarga}
Es un tipo de polimorfismo donde varios métodos pueden tener el mismo nombre si el tipo de parámetro que reciben o el número de ellos es distinto. 

\begin{center}
\begin{tabular}{ll}
Write(int i); & Escribe un entero\\
Write(long I); & Escribe un long\\
Write(float f); & Escribe un flotante\\
Write(string s); & Escribe una cadena de texto\\
\end{tabular}
\end{center}
\end{frame}


\begin{frame}[label={sec:org19b1a4d}]{Constructores y destructores}
Para utilizar un objeto hay que construirlo o crearlo y para ello se usa el \emph{Constructor de la Clase}. Para ello dependiendo del lenguaje existen dos procedimientos. 

\begin{enumerate}
\item Utilizando un método especial usando una \alert{palabra reservada}
\item Usando un operador especial normalmente \alert{new}
\end{enumerate}

Los \alert{destructores} hacen justo lo contrario que los constructores. 


La mayoría de los lenguajes tienen un constructor y un destructor \alert{por defecto}.
\end{frame}

\begin{frame}[label={sec:orgb762c82}]{Accesibilidad de Datos y Métodos}
Indican la \alert{visibilidad} que tiene una variable o método. 

\begin{block}{Públicos}
Son visibles dentro y fuera de la clase

Usan la palabra reservada \alert{public}
\end{block}

\begin{block}{Protegidos}
Son visibles dentro de su clase y las clases heredadas.

Usan la palabra reservada \alert{protected}
\end{block}


\begin{block}{Privados}
Solo son accesibles desde dentro de la clase en la que existen 

Usan la palabra reservada \alert{private} 



Lo normal es ver \alert{Métodos públicos} y \alert{propiedades protegidas o privadas}. 
\end{block}
\end{frame}
\end{document}
